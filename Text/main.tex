%!TEX TS-program = xelatex

% Шаблон документа LaTeX создан в 2018 году
% Алексеем Подчезерцевым
% В качестве исходных использованы шаблоны
% 	Данилом Фёдоровых (danil@fedorovykh.ru) 
%		https://www.writelatex.com/coursera/latex/5.2.2
%	LaTeX-шаблон для русской кандидатской диссертации и её автореферата.
%		https://github.com/AndreyAkinshin/Russian-Phd-LaTeX-Dissertation-Template

\documentclass[a4paper,14pt]{article}

\input{data/preambular.tex}
\begin{document} % конец преамбулы, начало документа
\input{data/title.tex}

\section{Постановка задачи}

Необходимо написать игру, в жанре пошаговой стратегии (Реального времени?) на 2 человека (бота?), в которой будет использоваться следующие структуры: классы, наследование, перегрузка операторов, шаблоны и полиморфизм. 

\section{Метод решения}

На игровом поле до начала игры будут размещены постройки, выполняющие одну из следующих функций: добыча ресурсов, производство дружественных юнитов, атака ближайший вражеских юнитов. Для реализации юнитов и зданий будут созданы классы, наследуемые от базовых (см. \imref{im:UML}) и реализующие механику игры "Камень-ножницы-бумага" путем настройки коэффициентов для совершения базовых действий. В процессе игры здания могут разрушатся и продолжать занимать ячейку, юниты же исчезают с игрового поля в случае гибели. Игра продолжается до тех пор, пока один из игроков полностью не уничтожит все владения противника.
Для реализации данной механики будет создан один общий класс игры, которому принадлежат классы игроков и общая информация по карте. В классах игроков содержится информацию об игроке, его юнитах и зданиях. Информация об основных игровых единицах представлена на стр. \pageref{im:UML}.

\section{Структура классов}

\begin{figure}[H]
	\centering
	\caption{Структура классов}
	\includegraphics[width=\linewidth]{class_diagramm.pdf}	
	\label{im:UML}
\end{figure}

\end{document} % конец документа

